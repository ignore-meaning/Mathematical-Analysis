\documentclass{ctexart}

\usepackage[x11names]{xcolor}   % 颜色
\usepackage{amsmath, amssymb}   % 数学公式与符号
\usepackage{hyperref}           % 超链接
\usepackage{graphicx}           % 插图
\usepackage{lmodern, bm}        % 额外字体、粗体样式
\usepackage[scr]{rsfso}         % mathscr字体
\usepackage{extarrows}          % 长等号
\usepackage{calc}               % 支持不同单位的长度计算
\usepackage{array}              % 表格
\usepackage{geometry}           % 页边距
\usepackage{amsthm}             % prove 环境

\hypersetup{colorlinks = true}  % 超链接颜色

\newcommand{\Pblock}[1]{\fbox{\parbox{\textwidth-28pt}{#1}}}    % 带边框的垂直盒子

\makeatletter                   % 设置 “定义等号”
\newcommand*{\defeq}{\mathrel{\rlap{%
                    \raisebox{0.3ex}{$\m@th\cdot$}}%
                    \raisebox{-0.3ex}{$\m@th\cdot$}}%
                    =}
\makeatother

% 定义两个计数器:“定义” 和 “定理”
\newcounter{definition}[subsection]
\newcounter{subdefinition}[definition]
\newcounter{theorem}[subsection]
\newcounter{subtheorem}[theorem]

\NewDocumentEnvironment{definition}{o}{\IfNoValueTF{#1}{\stepcounter{definition}\textbf{定义\,\thedefinition.}}{\stepcounter{definition}\textbf{定义\,\thedefinition\,(#1).}}}{\vspace{9pt}}
\NewDocumentEnvironment{subdefinition}{o}{\IfNoValueTF{#1}{\stepcounter{subdefinition}\textbf{定义\,\thedefinition.\thesubdefinition.}}{\stepcounter{subdefinition}\textbf{定义\,\thedefinition.\thesubdefinition\,(#1).}}}{}

\NewDocumentEnvironment{theorem}{o}{\IfNoValueTF{#1}{\stepcounter{theorem}\textbf{定理\,\thetheorem.}}{\stepcounter{theorem}\textbf{定理\,\thetheorem\,(#1).}}}{}
\NewDocumentEnvironment{subtheorem}{o}{\IfNoValueTF{#1}{\stepcounter{subtheorem}\textbf{定理\,\thetheorem.\thesubtheorem.}}{\stepcounter{subtheorem}\textbf{定理\,\thetheorem.\thesubtheorem\,(#1).}}}{}

\title{数学分析}
\author{陈程}
\date{\today}

\begin{document}
\maketitle
\tableofcontents

\section{一些通用的数学概念与记号}
略
\section{准备工作}
\subsection{广义实数}
\begin{definition}
    定义广义实数 $\overline{\mathbb{R}} \defeq \mathbb{R} \cup \{+\infty, -\infty\}$.\footnote{这里先不给出 $+\infty, -\infty$ 的具体性质,只要明确 $+\infty, -\infty \notin \mathbb{R}$ 且 $+\infty \neq -\infty$ 即可}
\end{definition}\vspace{9pt}

\stepcounter{definition}
\begin{subdefinition}
    对于实数 $a$,正实数 $\delta$,
    \[U^\delta(a) \defeq \bigl\{x \bigm| \lvert x - a \rvert < \delta \bigr\} = (a-\delta, a+\delta)\]
\end{subdefinition}

\begin{subdefinition}
    对于正实数 $\delta$,
    \[U^\delta(+\infty) \defeq (\delta , +\infty)\]
\end{subdefinition}

\begin{subdefinition}
    对于正实数 $\delta$,
    \[U^\delta(-\infty) \defeq (-\infty, -\delta)\]
\end{subdefinition}

\begin{subdefinition}
    对于广义实数 $a$,正实数 $\delta$,
    \[\mathring{U}^\delta(a) \defeq U^\delta(a) \backslash \{a\}\]
\end{subdefinition}

\begin{subdefinition}
    对于数集\footnote{实数集的子集} $E$、实数 $a$ 和正实数 $\delta$,
    \[U_E^\delta(a) \defeq U^\delta(a) \cap E\]
\end{subdefinition}

\begin{subdefinition}
    对于数集 $E$、实数 $a$ 和正实数 $\delta$,
    \[\mathring{U}_E^\delta(a) \defeq \mathring{U}^\delta(a) \cap E\]
\end{subdefinition}\vspace{9pt}

\begin{definition}
    对于数集 $E$ 和广义实数 $a$,如果
    \[\forall \delta > 0: \mathring{U}_E^\delta(a) \neq \varnothing\]

    则称 $a$ 是 数集 $E$ 的\emph{极限点}.
\end{definition}

\subsection{函数}
\subsubsection{定义}
\begin{definition}
    设 $\odot$ 是实数上\footnote{该定义很容易推广,不过数学分析不会用到复数}的 $n(n \geqslant 1)$ 元运算,$f_1, f_2, \cdots, f_n$ 都是函数且 $\displaystyle \bigcap_{1 \leqslant i \leqslant n}D(f_i) \neq \varnothing$. $\odot(f_1, f_2, \cdots, f_n)$ 是由下式
    \[D\bigl(\odot(f_1, f_2, \cdots, f_n)\bigr) = \bigcap_{1 \leqslant i \leqslant n}D(f_i)\]
    \[\forall x \in D\bigl(\odot(f_1, f_2, \cdots, f_n)\bigr): \odot(f_1, f_2, \cdots, f_n)(x) = \odot\bigl(f_1(x), f_2(x), \cdots, f_n(x)\bigr)\]
    所确定的函数. 因此 $\odot$ 也是函数上的 $n$ 元运算.
\end{definition}\vspace{9pt}

\begin{definition}
    设 $\odot$ 是函数上的 $n(n \geqslant 1)$ 元运算,$\mathcal{A}_1, \mathcal{A}_2, \cdots, \mathcal{A}_n$ 都是函数集(由函数组成的集合).
    \[\odot(\mathcal{A}_1, \mathcal{A}_2, \cdots, \mathcal{A}_n) \defeq \bigl\{ \odot(f_1, f_2, \cdots, f_n) \bigm| f_i \in \mathcal{A}_i, i=1,2,\cdots,n\bigr\}\]

    因此 $\odot$ 也是函数集上的 $n$ 元运算.
\end{definition}\vspace{9pt}

\begin{definition}
    设 $\odot$ 是函数集上的运算,这里以二元运算为例,对于函数 $f$ 和函数集 $\mathcal{A}$,
    \[\odot(f,\mathcal{A}) \defeq \odot(\{f\},\mathcal{A})\]
    \[\odot(\mathcal{A},f) \defeq \odot(\mathcal{A}, \{f\})\]

    因此 $\odot$ 也是函数与函数集的运算.
\end{definition}

\subsubsection{基本定理}
\begin{theorem}
    对于函数集 $\mathcal{A}, \mathcal{B}$,
    \[\mathcal{A} + \mathcal{B} = \mathcal{B} + \mathcal{A}\]
    \[\mathcal{AB} = \mathcal{BA}\]
\end{theorem}\vspace{9pt}

\begin{theorem}
    对于函数集 $\mathcal{A},\mathcal{B},\mathcal{C}$,
    \[\mathcal{A} + (\mathcal{B} + \mathcal{C}) = (\mathcal{A} + \mathcal{B}) + \mathcal{C} \tag{1}\]
    \[\mathcal{A}(\mathcal{BC}) = (\mathcal{AB})\mathcal{C} \tag{2}\]
    \[(\mathcal{A} + \mathcal{B}) \mathcal{C} \subseteq \mathcal{AC} + \mathcal{BC} \tag{3}\]
    \[(\mathcal{A} + \mathcal{B}) \circ \mathcal{C} \subseteq \mathcal{A}\circ\mathcal{C} + \mathcal{B}\circ\mathcal{C} \tag{4}\]
    \[\mathcal{A}\circ(\mathcal{B} + \mathcal{C}) \subseteq \mathcal{A} \circ \mathcal{B} + \mathcal{A} \circ \mathcal{C} \tag{5}\]
\end{theorem}
\section{极限}
\subsection{数列的极限}
\subsubsection{定义}
\begin{definition}
    定义域为正整数集 $\mathbb{N}^*$ 的函数被称为\emph{数列}\footnote{数列的值域未必是实数域的子集,但在数学分析中若无特别说明默认讨论的是实值函数与实值数列(值域是实数域的子集)}. 对于数列 $f$,$f(n)$ 被称为数列 $f$ 的\emph{第 $n$ 项},如果该数列的因变量是 $x$,那么也常把 $f(n)$ 记作 $x_n$.

    \textbf{补:}如果想强调某一数列(函数)的因变量,有时也用 $f_{*x}$ 来表示以 $x$ 作为因变量的数列 $f$.
\end{definition}\vspace{9pt}

\begin{definition}
    定义 $\lim$ 是一个关系,其\d{部}\d{分}性质由下面给出:

    对于函数 $f$,$(f,R) \in \lim \Rightarrow R \in \overline{\mathbb{R}}$.

    对于数列 $f$,广义实数 $R$,
    \[(f,R) \in \lim \Leftrightarrow \forall \varepsilon > 0: \exists N \in \mathbb{N}^*: \forall n > N: f(n) \in U^\varepsilon(R)\]

    可以证明,$\lim$ 对于数列而言是一个函数(实际上是一个泛函)\footnote{只要证明对于一个数列 $f$,$(f,R) \in \lim \wedge (f,R') \in \lim \Rightarrow R = R'$ 即可}. 对于一个数列 $f$,若 $f \in D(\lim)$,则称 $\lim(f)$ 为数列 $f$ 的\emph{极限},如果还满足 $\lim(f) \in \mathbb{R}$,则称数列 $f$ \emph{收敛}. 若数列 $f$ 不收敛,则称它\emph{发散}.

    \textbf{补:}对于数列 $f$,$\lim(f)$ 也常记作 $\displaystyle \lim_{n \rightarrow \infty} f(n)$.
\end{definition}\vspace{9pt}

\begin{definition}
    对于数列 $f$,如果
    \[\forall \varepsilon > 0: \exists N \in \mathbb{N}^*: \forall n,m > N: \bigl\vert f(m) - f(n)\bigr\vert < \varepsilon\]
    则称数列 $f$ 为\emph{柯西数列}.
\end{definition}\vspace{9pt}

\begin{definition}
    设 $f$ 是一个数列,数列 $g$ 的值域是正整数集的一个子集且 $g$ 严格递增,可以证明 $f\circ g$ 也是一个数列,称数列 $f \circ g$ 为数列 $f$ 的一个\emph{子列}.
\end{definition}\vspace{9pt}

\begin{definition}
    如果一个数列的某个子列有极限 $A \in \overline{\mathbb{R}}$,则称 $A$ 是该数列的一个\emph{部分极限}.
\end{definition}\vspace{9pt}

\begin{definition}
    对于数列 $f$,
    \[\sum_{n=1}^{\infty} f(n) \defeq \lim_{n \rightarrow \infty} \sum_{i=1}^{n} f(i)\]
\end{definition}\vspace{9pt}

\begin{definition}
    对于数列 $f$,构造数列 $g$ 满足
    \[\forall n\in \mathbb{N}^*: g(n) = \sum_{i=1}^{n} f(i)\]
    
    称构造出的数列 $g$ 是数列 $f$ 的\emph{级数}或\emph{无穷级数}.
\end{definition}\vspace{9pt}

\begin{definition}
    对于数列 $f_{*x}$,如果 $\displaystyle \sum_{n=1}^{\infty} \lvert x_n\rvert$ 是实数,那么称数列 $f$ 的级数\emph{绝对收敛}.
\end{definition}

\subsubsection{基本定理}
\begin{theorem}
    对于收敛数列 $f,g$,
    \[\lim(f+g) = \lim(f) + \lim(g)\]
    \[\lim(f \cdot g) = \lim(f) \cdot \lim(g)\]

    如果还有 $\forall n \in \mathbb{N}^*: g(n) \neq 0$ 以及 $\lim(g) \neq 0$,那么
    \[\lim(f/g) = \lim(f)/\lim(g)\]
\end{theorem}\vspace{9pt}

\begin{theorem}
    对于收敛数列 $f,g$,如果 $\lim(f) < \lim(g)$,那么
    \[\exists N \in \mathbb{N}^*: \forall n > N: f(n) < g(n)\]
\end{theorem}\vspace{9pt}

\begin{theorem}
    对于收敛数列 $f,g$,如果
    \[\exists n \in \mathbb{N}^*: \forall n > N: f(n) \geqslant g(n)\]

    那么
    \[\lim(f) \geqslant \lim(g)\]
\end{theorem}\vspace{9pt}

\begin{theorem}
    设数列 $f,g,h$ 满足
    \[\exists n \in \mathbb{N}^*: \forall n > N: f(n) \geqslant g(n) \geqslant h(n)\]
    \[\lim(f) = \lim(h)\]
    
    那么数列 $g$ 的极限也存在并且 $\lim(f) = \lim(g) = \lim(h)$.
\end{theorem}\vspace{9pt}

\begin{theorem}
    数列收敛的充要条件为它是柯西数列.
\end{theorem}\vspace{9pt}

\begin{theorem}
    不减数列收敛的充要条件是它上有界.
\end{theorem}\vspace{9pt}

\begin{theorem}
    每个数列都含有极限存在的子列.
\end{theorem}\vspace{9pt}

\begin{theorem}
    数列极限存在的充要条件是它的任意子列的极限都存在.
\end{theorem}\vspace{9pt}

\begin{theorem}
    一个数列的级数收敛的充分条件是该数列的级数绝对收敛.
\end{theorem}\vspace{9pt}

\begin{theorem}
    设对于数列 $f$,极限 $\displaystyle \lim_{n \rightarrow \infty} \left| \frac{f(n+1)}{f(n)}\right| = \alpha$ 存在,则以下命题成立:

    a) 如果 $\alpha < 1$,则 $f$ 的级数绝对收敛.

    b) 如果 $\alpha > 1$,则 $f$ 的级数发散.
\end{theorem}\vspace{9pt}

\begin{theorem}
    如果数列 $f$ 是非负不增数列,那么 $f$ 的级数收敛的充要条件是
    \[\sum_{k=1}^{\infty} 2^k f(2^k)\]
    存在且是实数.
\end{theorem}\vspace{9pt}

\begin{theorem}
    \[p \in (1, +\infty) \Rightarrow \sum_{n=1}^{\infty} \frac{1}{n^p} \in \mathbb{R}\]
    \[p \in (-\infty, 1] \Rightarrow \sum_{n=1}^{\infty} \frac{1}{n^p} = +\infty\]
\end{theorem}

\subsubsection{有趣的推论}
\begin{theorem}
    \[\mathrm{e} = \left(1 + \frac{1}{n}\right)^n + \frac{\theta_n}{n}\]
    其中 $0 < \theta_n < 3$.
\end{theorem}\vspace{9pt}

\begin{theorem}
    \[\mathrm{e} = 1 + \frac{1}{0!} + \frac{1}{1!} + \frac{1}{2!} + \cdots + \frac{1}{n!} + \frac{\theta_n}{n!n}\]
    其中 $0 < \theta_n < 1$.
\end{theorem}\vspace{9pt}

\begin{theorem}
    \[\mathrm{e} = \lim_{n \rightarrow \infty} \frac{n}{\sqrt[n]{n!}}\]
\end{theorem}\vspace{9pt}

\begin{theorem}
    若
    \[\forall n \in \mathbb{N}^*: y_{n+1} > y_n\]
    \[\lim_{n \rightarrow \infty} y_n = +\infty\]
    \[\lim_{n \rightarrow \infty} \frac{x_{n+1} - x_n}{y_{n+1} - y_n} = A \in \mathbb{R}\]

    那么
    \[\lim_{n \rightarrow \infty} \frac{x_n}{y_n} = A\]
\end{theorem}

\subsection{函数的极限}
\subsubsection{定义}
\begin{definition}
    对于函数 $f$ 和数集 $E$,$f|_E$ 被称为函数 $f$ 在集合 $E$ 上的\emph{限制},是由以下各式
    \[D(f|_E) = D(f) \cap E \neq \varnothing\]
    \[\forall x \in D(f|_E): f|_E(x) = f(x)\]
    所确定的函数.
\end{definition}\vspace{9pt}

\begin{definition}
    $\lim$ 的\d{部}\d{分}性质由下面给出:
    
    对于函数 $f$ 和广义实数 $a$,如果 $\bigl((f,a),R\bigr) \in \lim$,那么 $a$ 是函数 $f$ 定义域的极限点,$R \in \overline{\mathbb{R}}$.

    对于函数 $f$ 和其定义域 $E$ 的极限点 $a$,广义实数 $R$,
    \[\bigl((f,a),R\bigr) \in \lim \Leftrightarrow \forall \varepsilon > 0: \exists \delta > 0: \forall x \in \mathring{U}_E^\delta(a): f(x) \in U^\varepsilon(R)\]

    同样可以证明,$\lim$ 对于函数与广义实数所组成的序偶也是一个函数. 对于函数 $f$ 和广义实数 $a$,若 $(f,a) \in D(\lim)$,则称 $\lim(f,a)$ 为函数 $f$ 在 $a$ 处的\emph{极限}.

    \textbf{补:}对于函数 $f$ 和 广义实数 $a$,$\lim(f,a)$ 也常记作 $\displaystyle \lim_{x \rightarrow a} f(x)$.

    \textbf{补:}对于函数 $f$ 和 实数 $a$,$\lim(f|_{(a,+\infty)},a)$ 也常记作 $\lim(f,a^+)$ 或 $\displaystyle \lim_{x \rightarrow a^+} f(x)$. 类似地定义 $\lim(f,a^-)$ 和 $\displaystyle \lim_{x \rightarrow a^-} f(x)$.
\end{definition}\vspace{9pt}

\begin{definition}
    对于在数集 $E$ 上有定义的函数 $f$,称
    \[\omega(f;E) \defeq \sup_{x',x'' \in E} \bigl(f(x') - f(x'')\bigr) \defeq \sup \bigl\{f(x') - f(x'') \bigm| x',x'' \in E\bigr\}\]
    为函数 $f$ 在 $E$ 上的\emph{振幅}.
\end{definition}\vspace{9pt}

\begin{definition}
    对于函数 $f$ 和实数 $a$,如果 $a$ 是 $D(f)$ 的极限点,则
    \[\omega(f;a) \defeq \lim_{\delta \rightarrow 0^+} \omega(f;U_{D(f)}^\delta)\]
    被称为函数 $f$ 在 $a$ 处的\emph{振幅}.
\end{definition}\vspace{9pt}

\stepcounter{definition}
\begin{subdefinition}
    对于函数 $g$ 和 广义实数 $a$,$\underset{x \rightarrow a}{o(g)}$ 是所有满足下列性质
    \[\lim_{x \rightarrow a} \frac{f(x)}{g(x)} = 0\]
    的函数 $f$ 所组成的集合,在已明确 $a$ 的值的情况下也可以简写为 $o(g)$. 
\end{subdefinition}

\begin{subdefinition}
    对于函数集 $\mathcal{A}$ 和广义实数 $a$,
    \[\underset{x \rightarrow a}{o(\mathcal{A})} \defeq \bigcup_{g \in \mathcal{A}}\underset{x \rightarrow a}{o(g)}\]
\end{subdefinition}\vspace{9pt}

\stepcounter{definition}
\begin{subdefinition}
    对于函数 $g$ 和 广义实数 $a$,$\underset{x \rightarrow a}{O(g)}$ 是所有满足下列性质
    \[\exists \delta,M > 0: \forall x \in \mathring{U}_{D(g)}^\delta(a): \left|\frac{f(x)}{g(x)}\right| < M\]
    的函数 $f$ 所组成的集合,在已明确 $a$ 的值的情况下也可以简写为 $o(g)$.
\end{subdefinition}

\begin{subdefinition}
    对于函数集 $\mathcal{A}$ 和广义实数 $a$,
    \[\underset{x \rightarrow a}{O(\mathcal{A})} \defeq \bigcup_{g \in \mathcal{A}} \underset{x \rightarrow a}{O(g)}\]
\end{subdefinition}\vspace{9pt}

\begin{definition}
    对于函数 $f,g$ 和广义实数 $R$,如果
    \[\lim(f/g,R) = 1\]
    则称函数 $f$ 在趋近于 $R$ 时等价于函数 $g$,记作 $f \overset{x \rightarrow R}{\sim} g$ 或简记为 $f \sim g$.
\end{definition}

\subsubsection{基本定理}
\begin{theorem}
    设 $f,g$ 在 $a \in \overline{\mathbb{R}}$ 处的极限都存在且是实数,$a$ 是 $D(f) \cap D(g)$ 的极限点,有
    \[\lim(f+g,a) = \lim(f,a) + \lim(g,a)\]
    \[\lim(f \cdot g, a) = \lim(f,a) \cdot \lim(g,a)\]
    
    如果还有 $a$ 是 $D(f/g)$ 的极限点而且 $\lim(g,a) \neq 0$,那么
    \[\lim(f/g,a) = \frac{\lim(f,a)}{\lim(g,a)}\]
\end{theorem}\vspace{9pt}

\begin{theorem}
    对于函数 $f,g$ 和广义实数 $a$,如果 $\lim(f,a) < \lim(g,a)$,那么令 $E = D(f) \cap D(g)$ 便有
    \[\exists \delta > 0: \forall x \in \mathring{U}_E^\delta(a): f(x) < g(x)\]
\end{theorem}\vspace{9pt}

\begin{theorem}
    对于函数 $f,g$ 和 广义实数 $a$,如果两个函数在 $a$ 处的极限存在并且令 $E = D(f) \cap D(g)$ 便有
    \[\exists \delta > 0: \forall x \in \mathring{U}_E^\delta(a): f(x) \leqslant g(x)\]

    那么
    \[\lim(f,a) \leqslant \lim(g,a)\]
\end{theorem}\vspace{9pt}

\begin{theorem}
    对于函数 $f$ 和 $D(f)$ 的一个极限点 $a$,函数 $f$ 在 $a$ 处的极限存在且是实数当且仅当
    \[\forall \varepsilon > 0: \exists \delta > 0: \omega\Bigl(f,\mathring{U}_{D(f)}^\delta\Bigr) < \varepsilon\]
\end{theorem}\vspace{9pt}

\begin{theorem}
    设 $f$ 是 $X$ 到 $Y \subseteq \mathbb{R}$ 的映射,$g$ 是 $Y$ 到 $\mathbb{R}$ 的映射,
    \[\lim(g \circ f,a) = \lim\bigl(g, \lim(f,a)\bigr)\]
    成立的充分条件是(在这些极限都存在的情况下考虑)
    \[\exists \delta > 0: \forall x \in \mathring{U}_X^\delta: f(x) \neq \lim(f,a)\]
    或者
    \[\lim(f,a) \in \mathbb{R} \wedge \omega\bigl(g;\lim(f,a)\bigr) = 0\]
\end{theorem}\vspace{9pt}

\begin{theorem}
    对于定义域为 $E$ 的不减函数 $f$,$f$ 在 $\sup E$ 处存在有限的极限的充要条件是函数 $f$ 有上界.
\end{theorem}\vspace{9pt}

\begin{theorem}
    对于函数集 $\mathcal{A}, \mathcal{B}$,非零实数 $A$,
    \[Ao(\mathcal{A}) = o(\mathcal{A}) \tag{1}\]
    \[o(\mathcal{A}) \pm o(\mathcal{A}) \subseteq o(\mathcal{A}) \tag{2}\]
    \[o(\mathcal{A}) \cdot \mathcal{B} \subseteq o(\mathcal{AB}) \tag{3}\]
    \[o(\mathcal{A}) \circ \mathcal{B} \subseteq o(\mathcal{A} \circ \mathcal{B}) \tag{4}\]
    \[o(\mathcal{A} + \mathcal{B}) \subseteq o(\mathcal{A}) + o(\mathcal{B}) \tag{5}\]
    \[o(o(\mathcal{A})) \subseteq o(\mathcal{A}) \tag{6}\]
\end{theorem}\vspace{9pt}

\begin{theorem}
    对于函数集 $\mathcal{A}, \mathcal{B}$,非零实数 $A$,
    \[AO(\mathcal{A}) = O(\mathcal{A}) \tag{1}\]
    \[O(\mathcal{A}) \pm O(\mathcal{A}) \subseteq O(\mathcal{A}) \tag{2}\]
    \[O(\mathcal{A}) \cdot \mathcal{B} \subseteq O(\mathcal{AB}) \tag{3}\]
    \[O(\mathcal{A}) \circ \mathcal{B} \subseteq O(\mathcal{A} \circ \mathcal{B}) \tag{4}\]
    \[O(\mathcal{A} + \mathcal{B}) \subseteq O(\mathcal{A}) + O(\mathcal{B}) \tag{5}\]
    \[O(O(\mathcal{A})) \subseteq O(\mathcal{A}) \tag{6}\]
\end{theorem}

\subsubsection{有趣的推论}
\begin{theorem}
    当 $x \rightarrow 0$ 时,
    \[o(x^m) \circ \bigl[Ax^n + o(x^n)\bigr] \subseteq o(x^{mn}) \tag{1}\]
    \[x^m \circ \bigl[Ax^n + o(x^n)\bigr] \subseteq Ax^{mn} + o(x^{mn}) \tag{2}\]
    
    特别地,
    \[\bigl[P_n(x) + o(x^n)\bigr] \circ \bigl[Q_m(x) + o(x^m)\bigr] \subseteq \big[P_n(x) \circ Q_n(x)\big] + \big[o(x^n) + o(x^m)\big] \tag{3}\]
\end{theorem}

\subsubsection{例题}
\textbf{1.} 求极限
\[\lim_{x \rightarrow 0} \frac{\displaystyle \sqrt[7]{1 + \frac{x^2 - x^3}{1 + x^3}} - \cos x}{x^2}\]
\begin{proof}[解]
    已知当 $x \rightarrow 0$ 时
    \[\sqrt[7]{1 + x} \in 1 + \frac{1}{7}x + o(x)\]
    \[\frac{x^2-x^3}{1+x^3} \in x^2 + o(x^2)\]
    \[\cos x \in 1 - \frac{1}{2}x^2 + o(x^2)\]

    所以
    \begin{align*}
        \sqrt[7]{1 + \frac{x^2 - x^3}{1 + x^3}} &= \sqrt[7]{1 + x} \circ \frac{x^2-x^3}{1+x^3}\\
        &\in \left[1 + \frac{1}{7}x + o(x)\right] \circ \big[x^2 + o(x^2)\big]\\
        &\subseteq\footnotemark 1 + \frac{1}{7} x \circ \big[x^2 + o(x^2)\big] + o(x) \circ \big[x^2 + o(x^2)\big]\\
        &\subseteq\footnotemark 1 + \frac{1}{7}\big[x^2 + o(x^2)\big] + o(x^2)\\
        &\subseteq\footnotemark 1 + \frac{1}{7}x^2 + o(x^2)
    \end{align*}
    \footnotetext[6]{根据 2.2.2 节定理 2 式 $(4)$}
    \footnotetext[7]{根据 3.2.3 节定理 9 式 $(1),(2)$}
    \footnotetext[8]{根据 3.2.2 节定理 7 式 $(1),(2)$}
    \[\sqrt[7]{1 + \frac{x^2 - x^3}{1 + x^3}} - \cos x \in \left[1 + \frac{1}{7}x^2 + o(x^2)\right] - \left[1 - \frac{1}{2}x^2 + o(x^2)\right] \subseteq \frac{9}{14}x^2 + o(x^2)\]
    \[\frac{\displaystyle \sqrt[7]{1 + \frac{x^2 - x^3}{1 + x^3}} - \cos x}{x^2} \in \frac{\displaystyle \frac{9}{14}x^2 + o(x^2)}{x^2} \subseteq \frac{9}{14} + o(1)\]

    从而
    \[\lim_{x \rightarrow 0} \frac{\displaystyle \sqrt[7]{1 + \frac{x^2 - x^3}{1 + x^3}} - \cos x}{x^2} = \frac{9}{14}\]
\end{proof}
\section{连续函数}
\subsection{定义}
\begin{definition}
    对于定义域为 $E$ 的函数 $f$ 和 $E$ 上的一个极限点 $a \in E$,如果
    \[\lim_{x \rightarrow a} f(x) = f(a)\]
    \[\lim_{x \rightarrow a^-} f(x) = f(a)\]
    \[\lim_{x \rightarrow a^+} f(x) = f(a)\]
    则分别称函数 $f$ 在 $a$ 处\emph{连续},\emph{左连续},\emph{右连续}.
\end{definition}\vspace{9pt}

\begin{definition}
    对于在数集 $E$ 上有定义的函数 $f$,如果它在 $E$ 的每个点都连续则称函数 $f$ 在 $E$ 上\emph{连续}.
\end{definition}

\subsection{基本定理}
\begin{theorem}
    对于拥有相同定义域的函数 $f,g$,如果两个函数都在 $a$ 处连续,则 $f+g$、$f \cdot g$ 以及 $f/g$(当然要 $g(a) \neq 0$)也都在 $a$ 处连续.
\end{theorem}\vspace{9pt}

\begin{theorem}
    设函数 $f$ 在闭区间 $E = [a,b]$ 上有定义且连续,如果 $f(a)f(b) < 0$,那么存在 $\xi \in E$ 使得 $f(\xi) = 0$.
\end{theorem}
%\section{草稿本}
\[x \rightarrow 0\]
\[(\sin x - x+ x^2) \in o(x)\]
\[\sin x + x^2 \in x + o(x)\]

\Pblock{
    \[F(x) - x \in o(x^2)\]
    \[F(x) \in x + o(x^2) \subseteq x + o(x)\]
    \[F(x) \in x + o(x)\]
}
\[f,o(g)\]
\[f + o(g) \defeq \{f + h \mid h \in o(g)\}\]

\[f,g,\mathcal{A},\mathcal{B}\]\vspace{9pt}
\[(f+g)\mathcal{A} \subseteq f\mathcal{A} + g\mathcal{A}\]
\[(\mathcal{A} + \mathcal{B})f = f\mathcal{A} + f\mathcal{B}\]

\[x + o(x) + \frac{1}{3}x + o(x)\]
\[\frac{4}{3}x + o(x)\]
\end{document}