\section{微分学}
\subsection{定义}
\begin{definition}
    对于函数 $f$ 和实数 $a$,
    \[f'(a) \defeq \lim_{x \rightarrow a} \frac{f(x)-f(a)}{x-a}\]
    被称为函数 $f$ 在 $a$ 处的\emph{一阶导数}或简称为\emph{导数}. 如果 $f'(a)$ 存在,则称函数 $f$ 在 $a$ 处\emph{一阶可导}或\emph{可导}.
\end{definition}\vspace{9pt}

\begin{definition}
    请假装这里定义了高阶导数,谢谢.
\end{definition}\vspace{9pt}

\begin{definition}
    对于函数 $f$ 和数集 $E$,如果函数 $f$ 在 $E$ 上有定义而且在 $E$ 上任意一点都 $n(n \geqslant 1)$ 阶可导,则称
\end{definition}

\begin{definition}
    对于函数 $f$ 和实数 $x_0$,如果 $x_0$ 是 $E = D(f)$ 上的极限点,而且
    \[\exists \delta > 0: \forall x \in \mathring{U}_E^\delta(x_0): f(x) \leqslant f(x_0)\]
    
    则称 $x_0$ 是函数 $f$ 的\emph{极大值点},$f(x_0)$ 是函数 $f$ 的\emph{极大值}. 类似地定义\emph{极小值点}与\emph{极小值}. 极大值点与极小值点统称为\emph{极值点},极大值与极小值统称为\emph{极值}.
\end{definition}\vspace{9pt}

\begin{definition}
    对于函数 $f$ 的极值点 $x_0$,如果 $x_0$ 既是 $(x_0, +\infty) \cap D(f)$ 的极限点,又是 $(-\infty, x_0) \cap D(f)$ 的极限点,则称 $x_0$ 为函数 $f$ 的\emph{内极值点}.
\end{definition}

\subsection{基本定理}
\begin{theorem}
    对于函数 $f,g$,
    \[(f+g)' = f' + g'\]
    \[(fg)' = f'g + fg'\]
    \[(f/g)' = \frac{f'g - fg'}{g^2}\]
    \[(g \circ f)' = (g'\circ f) \cdot f'\]
\end{theorem}\vspace{9pt}

\begin{theorem}
    设函数 $f$ 在内极值点 $x_0$ 处可导,则 $f'(x_0) = 0$.
\end{theorem}\vspace{9pt}

\begin{theorem}
    如果函数 $f$ 在闭区间 $[a,b]$ 上有定义且连续,在开区间 $(a,b)$ 上可导,并且 $f(a) = f(b)$,则
    \[\exists \xi \in (a,b): f'(\xi) = 0\]
\end{theorem}\vspace{9pt}