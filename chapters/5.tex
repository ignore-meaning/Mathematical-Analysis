\section{微分学}
\subsection{定义}
\begin{definition}
    对于函数 $f$,$\Bigl(f^{(1)}\Bigr)$ 是由下式
    \[x \in D\Bigl(f^{(1)}\Bigr) \Leftrightarrow \lim(f,x) \in \mathbb{R} \wedge x \in \mathbb{R}\]
    \[\forall x \in D\Bigl(f^{(1)}\Bigr): f^{(1)}(x) = \lim(f,x)\]
    所确定的函数,被称为函数 $f$ 的\emph{一阶导函数}或简称为\emph{导函数}. 如果 $x \in D\Bigl(f^{(1)}\Bigr)$,则称函数 $f$ 在 $x$ 处\emph{一阶可导}或简称为\emph{可导}.
\end{definition}

\begin{definition}
    对于函数 $f$ 和正整数 $n$,
    \[f^{(n+1)} \defeq \Bigl(f^{(n)}\Bigr)'\]
    被称为函数 $f$ 的 $n+1$ 阶导函数. 如果 $x \in D\Bigl(f^{(n+1)}\Bigr)$,则称函数 $f$ 在 $x$ 处 \emph{$n+1$ 阶可导}.
\end{definition}\vspace{9pt}

\begin{definition}
    对于函数 $f$ 和数集 $E$,如果函数 $f$ 在 $E$ 上有定义而且在 $E$ 上任意一点都 $n(n \geqslant 1)$ 阶可导,则称函数 $f$ 在 $E$ 上 $n$ 阶可导.
\end{definition}\vspace{9pt}

\begin{definition}
    对于数集 $E$ 和正整数 $n$,$C^n(E)$ 表示所有在数集 $E$ 上 $n$ 阶可导的函数所组成的集合.
\end{definition}

\begin{definition}
    对于函数 $f$ 和实数 $x_0$,如果 $x_0$ 是 $E = D(f)$ 上的极限点,而且
    \[\exists \delta > 0: \forall x \in \mathring{U}_E^\delta(x_0): f(x) \leqslant f(x_0)\]
    
    则称 $x_0$ 是函数 $f$ 的\emph{极大值点},$f(x_0)$ 是函数 $f$ 的\emph{极大值}. 类似地定义\emph{极小值点}与\emph{极小值}. 极大值点与极小值点统称为\emph{极值点},极大值与极小值统称为\emph{极值}.
\end{definition}\vspace{9pt}

\begin{definition}
    对于函数 $f$ 的极值点 $x_0$,如果 $x_0$ 既是 $(x_0, +\infty) \cap D(f)$ 的极限点,又是 $(-\infty, x_0) \cap D(f)$ 的极限点,则称 $x_0$ 为函数 $f$ 的\emph{内极值点}.
\end{definition}\vspace{9pt}

\begin{definition}
    对于函数 $f$,实数 $x_0$,正整数 $n$,如果 $f$ 在 $x_0$ 处 $n$ 阶可导,则
    \[f(x_0) + \sum_{i=1}^{n} \frac{f^{(i)}(x_0)}{i!} \cdot (\mathrm{id} - x_0)^i\]
    被称为函数 $f$ 在 $x_0$ 处的 $n$ 阶泰勒函数.
\end{definition}

\subsection{基本定理}
\begin{theorem}
    对于函数 $f,g$,
    \[(f+g)' = f' + g'\]
    \[(fg)' = f'g + fg'\]
    \[(f/g)' = \frac{f'g - fg'}{g^2}\]
    \[(g \circ f)' = (g'\circ f) \cdot f'\]
\end{theorem}\vspace{9pt}

\begin{theorem}
    设函数 $f$ 在内极值点 $x_0$ 处可导,则 $f'(x_0) = 0$.
\end{theorem}\vspace{9pt}

\stepcounter{theorem}
\begin{subtheorem}
    如果函数 $f$ 在闭区间 $[a,b]$ 上有定义且连续,在开区间 $(a,b)$ 上可导,并且 $f(a) = f(b)$,则
    \[\exists \xi \in (a,b): f'(\xi) = 0\]
\end{subtheorem}

\begin{subtheorem}
    如果函数 $f$ 在闭区间 $[a,b]$ 上有定义且连续,在开区间 $(a,b)$ 上可导,则
    \[\exists \xi \in (a,b): f(b) - f(a) = f'(\xi) \cdot (b-a)\]
\end{subtheorem}

\begin{subtheorem}
    如果函数 $f,g$ 都在闭区间 $[a,b]$ 上有定义且连续,在开区间 $(a,b)$ 上可导,则
    \[\exists \xi \in (a,b): \frac{f(b) - f(a)}{g(b) - g(a)} = \frac{f'(\xi)}{g'(\xi)}\]
\end{subtheorem}\vspace{9pt}

\begin{theorem}
    函数 $f$ 在某一闭区间 $[a,b]$ 上函数值为某一固定常数的充要条件是 $f'$ 在开区间 $(a,b)$ 上有定义且函数值为零.
\end{theorem}