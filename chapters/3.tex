\section{极限}
\subsection{数列的极限}
\subsubsection{定义}
\begin{definition}
    定义域为正整数集 $\mathbb{N}^*$ 的函数被称为\emph{数列}\footnote{数列的值域未必是实数域的子集,但在数学分析中若无特别说明默认讨论的是实值函数与实值数列(值域是实数域的子集)}. 对于数列 $f$,$f(n)$ 被称为数列 $f$ 的\emph{第 $n$ 项},如果该数列的因变量是 $x$,那么也常把 $f(n)$ 记作 $x_n$.

    \textbf{补:}如果想强调某一数列(函数)的因变量,有时也用 $f_{*x}$ 来表示以 $x$ 作为因变量的数列 $f$.
\end{definition}\vspace{9pt}

\begin{definition}
    定义 $\lim$ 是一个关系,其\d{部}\d{分}性质由下面给出:

    设 $f$ 是一个数列,$(f,R) \in \lim \Rightarrow R \in \overline{\mathbb{R}}$.

    对于数列 $f$,广义实数 $R$,
    \[(f,R) \in \lim \Leftrightarrow \forall \varepsilon > 0: \exists N \in \mathbb{N}^*: \forall n > N: f(n) \in U^\varepsilon(R)\]

    可以证明,$\lim$ 对于数列而言是一个函数(实际上是一个泛函)\footnote{只要证明对于一个数列 $f$,$(f,R) \in \lim \wedge (f,R') \in \lim \Rightarrow R = R'$ 即可}. 对于一个数列 $f$,若 $f \in D(\lim)$,则称 $\lim(f)$ 为数列 $f$ 的\emph{极限},如果还满足 $\lim(f) \in \mathbb{R}$,则称数列 $f$ \emph{收敛}. 若数列 $f$ 不收敛,则称它\emph{发散}.

    \textbf{补:}对于数列 $f$,$\lim(f)$ 也常记作 $\displaystyle \lim_{n \rightarrow \infty} f(n)$.
\end{definition}\vspace{9pt}

\begin{definition}
    对于数列 $f$,如果
    \[\forall \varepsilon > 0: \exists N \in \mathbb{N}^*: \forall n,m > N: \bigl\vert f(m) - f(n)\bigr\vert < \varepsilon\]
    则称数列 $f$ 为\emph{柯西数列}.
\end{definition}\vspace{9pt}

\begin{definition}
    设 $f$ 是一个数列,数列 $g$ 的值域是正整数集的一个子集且 $g$ 严格递增,可以证明 $f\circ g$ 也是一个数列,称数列 $f \circ g$ 为数列 $f$ 的一个\emph{子列}.
\end{definition}\vspace{9pt}

\begin{definition}
    如果一个数列的某个子列有极限 $A \in \overline{\mathbb{R}}$,则称 $A$ 是该数列的一个\emph{部分极限}.
\end{definition}\vspace{9pt}

\begin{definition}
    对于数列 $f$,
    \[\sum_{n=1}^{\infty} f(n) \defeq \lim_{n \rightarrow \infty} \sum_{i=1}^{n} f(i)\]
\end{definition}\vspace{9pt}

\begin{definition}
    对于数列 $f$,构造数列 $g$ 满足
    \[\forall n\in \mathbb{N}^*: g(n) = \sum_{i=1}^{n} f(i)\]
    
    称构造出的数列 $g$ 是数列 $f$ 的\emph{级数}或\emph{无穷级数}.
\end{definition}\vspace{9pt}

\begin{definition}
    对于数列 $f_{*x}$,如果 $\displaystyle \sum_{n=1}^{\infty} \lvert x_n\rvert$ 是实数,那么称数列 $f$ 的级数\emph{绝对收敛}.
\end{definition}

\subsubsection{基本定理}
\begin{theorem}
    如果数列 $f,g$ 都收敛,则
    \[\lim(f+g) = \lim(f) + \lim(g)\]
    \[\lim(f \cdot g) = \lim(f) \cdot \lim(g)\]

    如果还有 $\forall n \in \mathbb{N}^*: g(n) \neq 0$ 以及 $\lim(g) \neq 0$,那么
    \[\lim(f/g) = \lim(f)/\lim(g)\]
\end{theorem}\vspace{9pt}

\begin{theorem}
    对于收敛数列 $f,g$,如果 $\lim(f) < \lim(g)$,那么
    \[\exists N \in \mathbb{N}^*: \forall n > N: f(n) < g(n)\]
\end{theorem}\vspace{9pt}

\begin{theorem}
    对于收敛数列 $f,g$,如果
    \[\forall n \in \mathbb{N}^*: f(n) \geqslant g(n)\]

    那么
    \[\lim(f) \geqslant \lim(g)\]
\end{theorem}\vspace{9pt}

\begin{theorem}
    如果数列 $f,g,h$ 满足条件:$\forall n \in \mathbb{N}^*: f(n) \geqslant g(n) \geqslant h(n)$ 而且数列 $f,h$ 有相同的极限,那么数列 $g$ 的极限也存在并且 $\lim(f) = \lim(g) = \lim(h)$.
\end{theorem}\vspace{9pt}

\begin{theorem}
    数列收敛的充要条件为它是柯西数列.
\end{theorem}\vspace{9pt}

\begin{theorem}
    不减数列收敛的充要条件是它上有界.
\end{theorem}\vspace{9pt}

\begin{theorem}
    每个数列都含有极限存在的子列.
\end{theorem}\vspace{9pt}

\begin{theorem}
    数列极限存在的充要条件是它的任意子列的极限都存在.
\end{theorem}\vspace{9pt}

\begin{theorem}
    一个数列的级数收敛的充分命题是该数列的级数绝对收敛.
\end{theorem}\vspace{9pt}

\begin{theorem}
    设对于数列 $f$,极限 $\displaystyle \lim_{n \rightarrow \infty} \left| \frac{f(n+1)}{f(n)}\right| = \alpha$ 存在,则以下命题成立:

    a) 如果 $\alpha < 1$,则 $f$ 的级数绝对收敛.

    b) 如果 $\alpha > 1$,则 $f$ 的级数发散.
\end{theorem}\vspace{9pt}

\begin{theorem}
    如果数列 $f$ 是非负不增数列,那么 $f$ 的级数收敛的充要条件是
    \[\sum_{k=1}^{\infty} 2^k f(2^k)\]
    存在且是实数.
\end{theorem}\vspace{9pt}

\begin{theorem}
    \[p \in (1, +\infty) \Rightarrow \sum_{n=1}^{\infty} \frac{1}{n^p} \in \mathbb{R}\]
    \[p \in (-\infty, 1] \Rightarrow \sum_{n=1}^{\infty} \frac{1}{n^p} = +\infty\]
\end{theorem}

\subsubsection{有趣的推论}
\begin{theorem}
    \[\mathrm{e} = \left(1 + \frac{1}{n}\right)^n + \frac{\theta_n}{n}\]
    其中 $0 < \theta_n < 3$.
\end{theorem}\vspace{9pt}

\begin{theorem}
    \[\mathrm{e} = 1 + \frac{1}{0!} + \frac{1}{1!} + \frac{1}{2!} + \cdots + \frac{1}{n!} + \frac{\theta_n}{n!n}\]
    其中 $0 < \theta_n < 1$.
\end{theorem}\vspace{9pt}

\begin{theorem}
    \[\mathrm{e} = \lim_{n \rightarrow \infty} \frac{n}{\sqrt[n]{n!}}\]
\end{theorem}\vspace{9pt}

\begin{theorem}
    若
    \[\forall n \in \mathbb{N}^*: y_{n+1} > y_n\]
    \[\lim_{n \rightarrow \infty} y_n = +\infty\]
    \[\lim_{n \rightarrow \infty} \frac{x_{n+1} - x_n}{y_{n+1} - y_n} = A \in \mathbb{R}\]

    那么
    \[\lim_{n \rightarrow \infty} \frac{x_n}{y_n} = A\]
\end{theorem}

\subsection{函数的极限}
\subsubsection{定义}
\begin{definition}
    对于函数 $f$ 和数集 $E$,$f|_E$ 被称为函数 $f$ 在集合 $E$ 上的\emph{限制},是由以下各式
    \[D(f|_E) = D(f) \cap E \neq \varnothing\]
    \[\forall x \in D(f|_E): f|_E(x) = f(x)\]
    所确定的函数.
\end{definition}\vspace{9pt}

\begin{definition}
    $\lim$ 的\d{部}\d{分}性质由下面给出:
    
    设 $a$ 是函数 $f$ 定义域的一个极限点,$\bigl((f,a),R\bigr) \in \lim \Rightarrow R \in \overline{\mathbb{R}}$.

    设 $a$ 是函数 $f$ 定义域 $E$ 的一个极限点,$R \in \overline{\mathbb{R}}$,
    \[\bigl((f,a),R\bigr) \in \lim \Leftrightarrow \forall \varepsilon > 0: \exists \delta > 0: \forall x \in \mathring{U}_E^\delta(a): f(x) \in U^\varepsilon(R)\]

    同样可以证明,$\lim$ 对于函数与函数定义域的极限点所组成的序偶也是一个函数. 对于一个函数 $f$ 和其定义域的极限点 $a$,若 $(f,a) \in D(\lim)$,则称 $\lim(f,a)$ 为函数 $f$ 在 $a$ 处的\emph{极限}.

    \textbf{补:}对于函数 $f$ 和 $f$ 定义域的极限点 $a$,$\lim(f,a)$ 也常记作 $\displaystyle \lim_{x \rightarrow a} f(x)$.

    \textbf{补:}对于函数 $f$ 和 $f$ 定义域的有限极限点 $a \in \mathbb{R}$,$\lim(f|_{(a,+\infty)},a)$(如果存在的话)也常记作 $\lim(f,a^+)$ 或 $\displaystyle \lim_{x \rightarrow a^+} f(x)$. 类似地定义 $\lim(f,a^-)$ 和 $\displaystyle \lim_{x \rightarrow a^-} f(x)$.
\end{definition}\vspace{9pt}

\begin{definition}
    对于在数集 $E$ 上有定义的函数 $f$,称
    \[\omega(f;E) \defeq \sup_{x',x'' \in E} \bigl(f(x') - f(x'')\bigr) \defeq \sup \bigl\{f(x') - f(x'') \bigm| x',x'' \in E\bigr\}\]
    为函数 $f$ 在 $E$ 上的\emph{振幅}.
\end{definition}\vspace{9pt}

\begin{definition}
    对于函数 $f$ 和函数定义域的一个极限点 $a$,$\omega(f;a)$ 被称为函数 $f$ 在 $a$ 处的\emph{振幅}(不一定存在),$\omega(f;a)$ 的定义见下:

    如果 $a \in \mathbb{R}$
    \[\omega(f;a) \defeq \lim_{\delta \rightarrow 0^+} \omega(f;U_{D(f)}^\delta)\]
    
    如果 $a \in \{+\infty, -\infty\}$
    \[\omega(f;a) \defeq \lim_{\delta \rightarrow +\infty} \omega(f;U_{D(f)}^\delta)\]
\end{definition}\vspace{9pt}

\stepcounter{definition}
\begin{subdefinition}
    对于函数 $g$ 和 $D(g)$ 的一个极限点 $a$,$\underset{x \rightarrow a}{o(g)}$ 是满足下列性质
    \[\lim_{x \rightarrow a} \frac{f(x)}{g(x)} = 0\]
    的函数 $f$ 所组成的集合. 
\end{subdefinition}

\begin{subdefinition}
    对于函数集 $\mathcal{A}$ 和广义实数 $a$,
    \[\underset{x \rightarrow a}{o(\mathcal{A})} \defeq \bigcup_{g \in \mathcal{A}}\underset{x \rightarrow a}{o(g)}\]
\end{subdefinition}\vspace{9pt}

\stepcounter{definition}
\begin{subdefinition}
    对于函数 $g$ 和 $D(g)$ 的一个极限点 $a$,$\underset{x \rightarrow a}{O(g)}$ 是满足下列性质
    \[\exists \delta,M > 0: \forall x \in \mathring{U}_{D(g)}^\delta(a): \left|\frac{f(x)}{g(x)}\right| < M\]
    的函数 $f$ 所组成的集合.
\end{subdefinition}

\begin{subdefinition}
    对于函数集 $\mathcal{A}$ 和广义实数 $a$,
    \[\underset{x \rightarrow a}{O(\mathcal{A})} \defeq \bigcup_{g \in \mathcal{A}} \underset{x \rightarrow a}{O(g)}\]
\end{subdefinition}\vspace{9pt}

\begin{definition}
    对于拥有相同定义域的函数 $f,g$ 和广义实数 $R$,如果
    \[\lim_{x \rightarrow R} \frac{f(x)}{g(x)} = 1\]
    则称函数 $f$ 在趋近于 $R$ 时等价于函数 $g$,记作 $f \overset{x \rightarrow R}{\sim} g$ 或简记为 $f \sim g$.
\end{definition}

\subsubsection{基本定理}
\begin{theorem}
    如果函数 $f,g$ 具有相同的定义域,而且 $\lim(f,a) = A \in \mathbb{R}, \lim(g,a) = B \in \mathbb{R}$,那么
    \[\lim(f+g,a) = A + B\]
    \[\lim(f \cdot g, a) = A \cdot B\]
    
    如果还有 $\forall x \in D(g): g(x) \neq 0$ 以及 $B \neq 0$,那么
    \[\lim(f/g,a) = \frac{A}{B}\]
\end{theorem}\vspace{9pt}

\begin{theorem}
    若函数 $f,g$ 拥有相同的定义域 $E$ 并且 $\lim(f,a) < \lim(g,a)$,那么
    \[\exists \delta > 0: \forall x \in \mathring{U}_E^\delta(a): f(x) < g(x)\]
\end{theorem}\vspace{9pt}

\begin{theorem}
    对于拥有相同定义域 $E$ 的函数 $f,g$ 和 $E$ 的一个极限点 $a$,如果两个函数在 $a$ 处的极限存在并且
    \[\exists \delta > 0: \forall x \in \mathring{U}_E^\delta(a): f(x) \leqslant g(x)\]

    那么
    \[\lim(f,a) \leqslant \lim(g,a)\]
\end{theorem}\vspace{9pt}

\begin{theorem}
    对于函数 $f$ 和 $D(f)$ 的一个极限点 $a$,函数 $f$ 在 $a$ 处的极限存在且是实数当且仅当
    \[\forall \varepsilon > 0: \exists \delta > 0: \omega(f,\mathring{U}_{D(f)}^\delta) < \varepsilon\]
\end{theorem}\vspace{9pt}

\begin{theorem}
    设 $f$ 是 $X$ 到 $Y \subseteq \mathbb{R}$ 的映射,$g$ 是 $Y$ 到 $\mathbb{R}$ 的映射,
    \[\lim(g \circ f,a) = \lim\bigl(g, \lim(f,a)\bigr)\]
    成立的充分条件是(在这些极限都存在的情况下考虑)
    \[\exists \delta > 0: \forall x \in \mathring{U}_X^\delta: f(x) \neq \lim(f,a)\]
    或者
    \[a \in \mathbb{R} \wedge \omega\bigl(g;\lim(f,a)\bigr) = 0\]
\end{theorem}\vspace{9pt}

\begin{theorem}
    对于定义域为 $E$ 的不减函数 $f$,$f$ 在 $\sup E$ 处存在有限的极限的充要条件是函数 $f$ 有上界.
\end{theorem}\vspace{9pt}

\begin{theorem}
    \[Ao(\mathcal{A}) = o(\mathcal{A}) \quad (A \neq 0) \tag{1}\]
    \[o(\mathcal{A}) \pm o(\mathcal{A}) \subseteq o(\mathcal{A}) \tag{2}\]
    \[o(\mathcal{A}) \cdot \mathcal{B} \subseteq o(\mathcal{AB}) \tag{3}\]
    \[o(\mathcal{A}) \circ \mathcal{B} \subseteq o(\mathcal{A} \circ \mathcal{B}) \tag{4}\]
    \[o(\mathcal{A} + \mathcal{B}) \subseteq o(\mathcal{A}) + o(\mathcal{B}) \tag{5}\]
    \[o(o(\mathcal{A})) \subseteq o(\mathcal{A}) \tag{6}\]
\end{theorem}\vspace{9pt}

\begin{theorem}
    \[O(f) = AO(f) \quad (A \neq 0) \tag{1}\]
    \[O(f) \pm O(f) \subseteq O(f) \tag{2}\]
    \[\mathcal{A} \cdot O(f) \subseteq O(\mathcal{A} \cdot f) \tag{3}\]
    \[O(f) \circ \mathcal{A} \subseteq O(f \circ \mathcal{A}) \tag{4}\]
    \[O(\mathcal{A} + \mathcal{B}) \subseteq O(\mathcal{A}) + O(\mathcal{B}) \tag{5}\]
    \[O(O(g)) \subseteq O(g) \tag{6}\]
\end{theorem}

\subsubsection{有趣的推论}
\begin{theorem}
    \[o(x^m) \circ \bigl[Ax^n + o(x^n)\bigr] \subseteq o(x^{mn}) \tag{1}\]
    \[x^m \circ \bigl[Ax^n + o(x^n)\bigr] \subseteq Ax^{mn} + o(x^{mn}) \tag{2}\]
    
    特别地,
    \[\bigl[P_n(x) + o(x^n)\bigr] \circ \bigl[Q_m(x) + o(x^m)\bigr] \subseteq \big[P_n(x) \circ Q_n(x)\big] + \big[o(x^n) + o(x^m)\big] \tag{3}\]
\end{theorem}

\subsubsection{例题}
\textbf{1.} 求极限
\[\lim_{x \rightarrow 0} \frac{\displaystyle \sqrt[7]{1 + \frac{x^2 - x^3}{1 + x^3}} - \cos x}{x^2}\]
\begin{proof}[解]
    已知当 $x \rightarrow 0$ 时
    \[\sqrt[7]{1 + x} \in 1 + \frac{1}{7}x + o(x)\]
    \[\frac{x^2-x^3}{1+x^3} \in x^2 + o(x^2)\]
    \[\cos x \in 1 - \frac{1}{2}x^2 + o(x^2)\]

    所以
    \begin{align*}
        \sqrt[7]{1 + \frac{x^2 - x^3}{1 + x^3}} &= \sqrt[7]{1 + x} \circ \frac{x^2-x^3}{1+x^3}\\
        &\in \big[1 + \frac{1}{7}x + o(x)\big] \circ \big[x^2 + o(x^2)\big]\\
        &\subseteq\footnotemark 1 + \frac{1}{7} x \circ \big[x^2 + o(x^2)\big] + o(x) \circ \big[x^2 + o(x^2)\big]\\
        &\subseteq\footnotemark 1 + \frac{1}{7}\big[x^2 + o(x^2)\big] + o(x^2)\\
        &\subseteq\footnotemark 1 + \frac{1}{7}x^2 + o(x^2)
    \end{align*}
    \[\sqrt[7]{1 + \frac{x^2 - x^3}{1 + x^3}} - \cos x \in \left[1 + \frac{1}{7}x^2 + o(x^2)\right] - \left[1 - \frac{1}{2}x^2 + o(x^2)\right] \subseteq \frac{9}{14}x^2 + o(x^2)\]
    \[\frac{\displaystyle \sqrt[7]{1 + \frac{x^2 - x^3}{1 + x^3}} - \cos x}{x^2} \in \frac{\displaystyle \frac{9}{14}x^2 + o(x^2)}{x^2} \subseteq \frac{9}{14} + o(1)\]

    从而
    \[\lim_{x \rightarrow 0} \frac{\displaystyle \sqrt[7]{1 + \frac{x^2 - x^3}{1 + x^3}} - \cos x}{x^2} = \frac{9}{14}\]
    \footnotetext[6]{根据 2.2.2 节定理 2 式 $(4)$}
    \footnotetext[7]{根据 3.2.3 节定理 9 式 $(1),(2)$}
    \footnotetext[8]{根据 3.2.2 节定理 7 式 $(1),(2)$}
\end{proof}