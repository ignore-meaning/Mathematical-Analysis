\section{连续函数}
\subsection{定义}
\begin{definition}
    对于定义域为 $E$ 的函数 $f$ 和 $E$ 的一个极限点 $a \in \mathbb{R}$,如果
    \[\lim_{x \rightarrow a} f(x) = f(a)\]
    \[\lim_{x \rightarrow a^-} f(x) = f(a)\]
    \[\lim_{x \rightarrow a^+} f(x) = f(a)\]
    则分别称函数 $f$ 在 $a$ 处\emph{连续},\emph{左连续},\emph{右连续}.
\end{definition}\vspace{9pt}

\begin{definition}
    对于在数集 $E$ 上有定义的函数 $f$,如果它在 $E$ 的每个点都连续,则称函数 $f$ 在 $E$ 上\emph{连续}.
\end{definition}\vspace{9pt}

\begin{definition}
    $C(E)$ 表示在所有在数集 $E$ 上连续的函数所组成的集合.
\end{definition}\vspace{9pt}

\begin{definition}
    如果函数 $f$ 在 $a$ 处不连续,则称 $a$ 是函数 $f$ 的\emph{间断点}.

    对于函数 $f$ 的间断点 $a$,如果 $\lim(f,a)$ 存在,则称 $a$ 是函数 $f$ 的\emph{可去间断点}.
\end{definition}\vspace{9pt}

\stepcounter{definition}
\begin{subdefinition}
    如果函数 $f$ 在 $a$ 处不连续但极限
    \[\lim_{x \rightarrow a^-} f(x),\quad \lim_{x \rightarrow a^+}f(x)\]
    都存在,则称 $a$ 是函数 $f$ 的\emph{第一类间断点}.
\end{subdefinition}

\begin{subdefinition}
    如果函数 $f$ 在 $a$ 处不连续但极限
    \[\lim_{x \rightarrow a^-}f(x), \quad \lim_{x \rightarrow a^+}f(x)\]
    至少有一个不存在,则称 $a$ 是函数 $f$ 的\emph{第二类间断点}.
\end{subdefinition}\vspace{9pt}

\begin{definition}
    对于函数 $f$ 与 数集 $E$,如果
    \[\forall \varepsilon > 0: \exists \delta > 0: \forall x', x'' \bigl(\lvert x' - x''\rvert < \delta\bigr): \lvert f(x') - f(x'')\rvert < \varepsilon\]
    那么称函数 $f$ 在数集 $E$ 上\emph{一致连续}.
\end{definition}

\subsection{基本定理}
\begin{theorem}
    对于拥有相同定义域的函数 $f,g$,如果两个函数都在 $a$ 处连续,则 $f+g$、$f \cdot g$ 以及 $f/g$(当然要 $g(a) \neq 0$)也都在 $a$ 处连续.
\end{theorem}\vspace{9pt}

\begin{theorem}
    设 $f:X \rightarrow Y, g:Y \rightarrow \mathbb{R}$,如果 $f$ 在 $a \in \mathbb{R}$ 处连续而且 $g$ 在 $f(a)$ 处连续,则 $g \circ f$ 在 $a$ 处连续.
\end{theorem}\vspace{9pt}

\begin{theorem}
    设函数 $f$ 在闭区间 $E = [a,b]$ 上有定义且连续,如果 $f(a)f(b) < 0$,那么存在 $\xi \in E$ 使得 $f(\xi) = 0$.
\end{theorem}\vspace{9pt}

\begin{theorem}
    在闭区间上连续的函数在该区间上有界. 闭区间上的连续函数取得到最大值也取得到最小值.
\end{theorem}\vspace{9pt}

\begin{theorem}
    在闭区间上连续的函数也在该区间上一致连续.
\end{theorem}\vspace{9pt}

\begin{theorem}
    连续函数的反函数(如果存在的话)也连续.
\end{theorem}

\subsection{有趣的推论}
\begin{theorem}
    如果 $a$ 是单调函数 $f$ 的间断点且 $a$ 不是函数 $f$ 定义域的端点,则极限
    \[\lim_{x \rightarrow a^-}f(x), \quad \lim_{x \rightarrow a^+}f(x)\]
    至少有一个存在.
\end{theorem}\vspace{9pt}

\begin{theorem}
    单调函数的间断点的集合至多可数.
\end{theorem}