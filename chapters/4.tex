\section{连续函数}
\subsection{定义}
\begin{definition}
    对于定义域为 $E$ 的函数 $f$ 和 $E$ 上的一个极限点 $a \in E$,如果
    \[\lim_{x \rightarrow a} f(x) = f(a)\]
    \[\lim_{x \rightarrow a^-} f(x) = f(a)\]
    \[\lim_{x \rightarrow a^+} f(x) = f(a)\]
    则分别称函数 $f$ 在 $a$ 处\emph{连续},\emph{左连续},\emph{右连续}(左右连续只考虑 $a \in \mathbb{R}$ 的情况).
\end{definition}\vspace{9pt}

\begin{definition}
    对于在数集 $E$ 上有定义的函数 $f$,如果它在 $E$ 的每个点都连续,则称函数 $f$ 在 $E$ 上\emph{连续}.
\end{definition}\vspace{9pt}

\begin{definition}
    $C(E)$ 表示在所有在数集 $E$ 上连续的函数所组成的集合.
\end{definition}\vspace{9pt}

\begin{definition}
    如果函数 $f$ 在 $a$ 处不连续,则称 $a$ 是函数 $f$ 的\emph{间断点}.

    对于函数 $f$ 的间断点 $a$,如果 $\lim(f,a)$ 存在,则称 $a$ 是函数 $f$ 的\emph{可去间断点},否则称 $a$ 是函数 $f$ 的\emph{不可去间断点}.
\end{definition}

\subsection{基本定理}
\begin{theorem}
    对于拥有相同定义域的函数 $f,g$,如果两个函数都在 $a$ 处连续,则 $f+g$、$f \cdot g$ 以及 $f/g$(当然要 $g(a) \neq 0$)也都在 $a$ 处连续.
\end{theorem}\vspace{9pt}

\begin{theorem}
    设函数 $f$ 在闭区间 $E = [a,b]$ 上有定义且连续,如果 $f(a)f(b) < 0$,那么存在 $\xi \in E$ 使得 $f(\xi) = 0$.
\end{theorem}